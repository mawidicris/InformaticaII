\documentclass{article}
\usepackage[utf8]{inputenc}

\title{Proycyo}
\author{mawidicris }
\date{March 2020}

\usepackage{natbib}
\usepackage{graphicx}

\begin{document}
\begin{center}
\includegraphics[scale=0.090]{Escudo-UdeA.svg.png}
\end{center}
\vspace{50pt}
\begin{center}
\bf{\sc\Large 'Los orígenes de la computación y su impacto hoy en día'}\\
\end{center}
\vspace{50pt}
\begin{center}
\begin{center}
\bf{\sc\large Por:}\\
\end{center}
\bf{\sc\large Cristian Daniel Padrón Hernández}\\
\end{center}
\begin{center}

\bf{\sc\large C.C: 1152717544}\\
\end{center}
\vspace{50pt}
\begin{center}
\bf{\sc\large Facultad de ingeniería}\\
\end{center}
\begin{center}
\bf{\sc\large Medellín}
\end{center}
\begin{center}
\bf{\sc\large 2020}\\
\end{center}\



\newpage
\Large
\section{Introducción}
 Hoy en día la tecnología y en general la computacion es parte fundamental, por no decir que ya es algo intrínseco de todos, es raro conocer a alguien que no haya por lo menos visto o usado un dispositivo electrónico; Pero eso nos hace preguntarnos ¿desde cuándo la tecnología ha influído tanto en nosotros? ¿desde cuando se ha vuelto tan importante en nosotros? 
\newline
Antes de respoder a esa preguntas es importante ir a origen de este fenomeno para poder saber desde qué punto impactó en nosotros, desde qué momento llegó para quedarse. 

\newpage
\Large
\section{Los orígenes de la computación y su impacto hoy en día}

Fue en 1874 cuando el matemático Greor Cantor dio inicio a la  llamada "crisis de los fundamentos", para entonces en la mente de muchos, en especial lo que se dedican a la ciencia había un unico pensamiento, las matemáticas son perfectas, únicas e irrefutables. Cantor planteó algo impensable, la filosfía y las matemáticas van de la mano, muchos pensarían que estaba loco, otros que solo quería llamar la atención, pero al final, el tiempo le dio la razón. 
\newline
En ese orden de ideas podemos decir que las matemáticas consideradas hasta ahora como el lenguaje de la naturaleza y que, parecen contener las leyes por las cuales se rige el universo se vieron retadas y cuestionadas de una forma tan drástica, y es que la crisis de los fundamentos como bien lo dice en su nombre no ponía en duda los fundamentos de las matemáticas como algo meramente académico e intelectual, sino que ponía en tela de juicio los fundamentos que han construido, moldeado y expresado nuestra sociedad como hoy la conocemos. Los teoremas de incompletitud de Gödel hicieron un hueco bastante profundo en las matemáticas, dos teoremas (en el que el segundo es un caso particular del primero) que a primera vista pueden parecer bastante simples (u obvios) hay que recalcar que su esencia tiene un desarrollo bastante elaborado, con una deducción exquisita y como no podía ser menos hablando de matemáticas, con toda la rigurosidad y formalidad del caso.
\newpage
El primero de ellos afirma que bajo ciertas condiciones, ninguna teoría matemática formal capaz de describir los números naturales y la aritmética con suficiente expresividad, es a la vez consistente y completa. Es decir, si los axiomas de dicha teoría no se contradicen entre sí, entonces existen enunciados que no se pueden probar ni refutar a partir de ellos. El segundo teorema de incompletitud afirma que una de las sentencias indecidibles de dicha teoría es aquella que «afirma» la consistencia de la misma. Es decir, que si el sistema de axiomas en cuestión es consistente, no es posible demostrarlo mediante dichos axiomas.Gödel descubrió y demostró que los axiomas, aquellas proposiciones atómicas y por ende indemostrables (que siempre se asume que son afirmaciones verdaderas) en su intento por abarcar todo el dominio de las matemáticas y afirmar o refutar cualquier conjetura a partir de su descomposición en los distintos axiomas habían dejado de lado aquellos enunciados que, por la misma consistencia de los axiomas se hace imposible afirmar o refutar dichos enunciados, en pocas palabras Gödel demostró que las matemáticas por su propia consistencia son inconsistentes en sí mismas, es algo que yo mismo denominaría como la “paradoja de la perfección y la imperfección” y es que el ser humano en su intento por crear sistemas perfectos olvida que la naturaleza es imperfecta, y que de esa imperfección es de donde nace su belleza. 

\newpage
Todo esto, nos lleva al nacimiento de la computación, y resaltando más importate cuando en 1936-1937 el matemático Alan Turing dio a conocer al mundo lo que se conoce como "La máquina universal de Turing", un prototipo de lo que serían más adelante las computadoras, desde entonces se ha avanzado mucho en este ámbito.
\newline
Lo que empezó con máquinas incluso más grandes que una persona hoy en día se puede llevar en el bolsillo, permitiendo controlar muchos aspectos en nuestra vida, nuestra rutina, nuestro trabajo, nuestra familia. Nos une y nos separa. Pero es algo que llegó para quedarse. 







\newpage

\section{Conclusión}
Hay que decir que la crisis de los fundamentos era una consecuencia inevitable en el desarrollo de las matemáticas, fue ese punto de inflexión que nos hacía falta para darnos cuenta de nuestros límites y saber que por más que lo intentemos hay cosas que no podemos cambiar y que hay una naturaleza imperfecta y caótica que nos domina, y en nuestro intento por dominarla nos desarrollamos y crecemos como civilización, cada día creamos y mejoramos dispositivos con mayor capacidad de cómputo para que resuelvan por nosotros esos problemas que nuestro primitivo cerebro no es capaz de descifrar, y es que eso es lo maravilloso de la ciencia, darse cuenta de sus errores y evolucionar a partir de ello, por eso es que hoy esos dispositivos de cómputo son tan necesarios, porque la ciencia que hoy conocemos no es más que un legado de muchos años de nuestra civilización humana y por ello es que cada vez surgen problemas más y más complejos que requieren de la asistencia de máquinas para poder ser resueltos, y lo maravilloso de todo esto es que aun con la capacidad de cómputo con la que dispone nuestra sociedad hoy en día aún hay problemas y conjeturas que siguen esperando una respuesta.


\newpage

\begin{thebibliography}{X}
\bibitem{Baz} \textsc{John T. Baldwin} y \textsc{Olivier Lessmann},
\textit{What is Russell's paradox?}, Recuperado de: https://www.scientificamerican.com/article/what-is-russells-paradox/,
1998
\bibitem{Baz}
\textit{Teoremas de incompletitud de Gödel}, Recuperado de: https://es.wikipedia.org/wiki/Teoremas_de_incompletitud_de_G%C3%B6del

\bibitem{Baz} \textsc{Joseph Warren Dauben} 
\textit{GEORG CANTOR: His Mathematics and Philosophy of the Infinite}, Recuperado de: https://math.dartmouth.edu/~matc/Readers/HowManyAngels/Cantor/Cantor.html, 1979

\end{thebibliography}
\end{document}
